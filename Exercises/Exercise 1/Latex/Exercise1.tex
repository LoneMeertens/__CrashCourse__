\documentclass[10pt,a4paper]{article}
\usepackage[utf8]{inputenc}
\usepackage{amsmath}
\usepackage{amsfonts}
\usepackage{amssymb}
\usepackage[left=2cm,right=2cm,top=2cm,bottom=2cm]{geometry}
\usepackage{url}
\usepackage{graphicx}
\usepackage[per-mode=symbol]{siunitx}
% Package for hyperlink, without ugly box around and nice blue color for text
\usepackage{hyperref}
\usepackage{xcolor}
\hypersetup{
	colorlinks,
	linkcolor={red!50!black},
	citecolor={blue!50!black},
	urlcolor={blue!80!black}
}
\setlength\parindent{0pt}

\usepackage[
    type={CC},
    modifier={by-sa},
    version={4.0},
]{doclicense}

\begin{document}
\title{Exercise 1: Exploring IDEAS}
\author{Filip Jorissen, Damien Picard, Jelger Jansen, Iago Cupeiro Figueroa}
\date{KU Leuven, September 23, 2022}
\maketitle


\doclicenseThis


\section*{Introduction}
The goal of this first exercise is to explore and become familiar with the Dymola environment and ... COMPLETE
Modelica and the IDEAS library. 
Since the IDEAS library components are typically used
by combining several components graphically, the use of 
equations falls outside of the scope of this exercise.\\

For this exercise you will create a model of a simple house,
consisting of a heating system, one building zone 
and a ventilation model. 
The exercise starts from a template file that should 
not produce any errors. This file will be extended in
several steps, adding complexity.
In between each step the user should be able to simulate the
model, i.e. no errors should be produced and simulation results 
may be compared.\\

Prerequisites are that you should have the latest version of Dymola
installed. You should have a working compiler, and a license. 
Dymola can be downloaded from 
\href{http://www.3ds.com/products-services/catia/products/dymola/trial-version/}{this link}. 
Installation instructions for Dymola and a C compiler can be found 
\href{http://www.3ds.com/fileadmin/PRODUCTS/CATIA/DYMOLA/PDF/Installation.pdf}{here}.
The latest version from the \href{https://github.com/open-ideas/IDEAS}{IDEAS library} should be downloaded and opened in Dymola. 
To verify your installation, try to simulate \path{IDEAS.Fluid.Actuators.Dampers.Examples.Damper} by opening the simulation tab (tab bar at the top) and by clicking \textit{Simulate}. Finally, download 
\textit{SimpleHouseTemplate.mo} from \url{Add_new_and_correct_link}
and load it into Dymola.\\

In the following sections the simple house model is discussed 
in several steps. The graphical representation of the final model is 
given in Figure~\ref{fig:simpleHouse}.
Each step first qualitatively explains the model part.
Secondly, the names of the required IDEAS models 
are listed.
Thirdly, we provide high-level instructions of how to
set up the model.
If these instructions are not clear immediately, 
have a look at the model documentation and at the type of
connectors the model has, 
try out some things, 
make an educated guess, etc.
Finally, we provide reference results that allow you to check
if your implementation is correct. 
Depending on the parameter values that you choose, results
may differ.
 



\section{Building wall model}
\paragraph{Qualitative discussion}
tekst

\paragraph{Required models}
In this first step only the Modelica Standard Library (MSL) 
model \path{Modelica.Thermal.HeatTransfer.Components.HeatCapacitor}
is required.

\paragraph{Connection instructions}
Connect the heat capacitor to the thermal resistor.

\paragraph{Reference result}
If you correctly added the model of the heat capacitor, connected it to the resistor and 
added the parameter values for $C$ and $R$
\footnote{Double-click on a component to see a list of its parameters. Gray values indicated default values.},
then you should be able to simulate the model.
To do this, go to the simulation tab (tab bar at the top),
open the simulation options 'Setup',
and set the model `Stop time' to 1e6 seconds.
You can now simulate the model.
You can plot individual variables values by clicking on their name in the variable browser on the left.
Now plot the wall capacitor temperature value `\textit{T}'. 
It should look like Figure~\ref{fig:res1}.




\end{document}